\section{PosgreSQL}

PostgreSQL es un gestor de base de datos relacional que puede correr tanto bajo
sistemas operativos Windows como en distribuciones Linux como Red Hat, Suse,
Centos, etc.

Como muchos otros proyectos de código abierto, el desarrollo de PostgreSQL
no es manejado por una empresa y/o persona, sino que es dirigido por una comunidad
de desarrolladores que trabajan de forma desinteresada, altruista, libre y/o
apoyados por organizaciones comerciales. Dicha comunidad es denominada
el PGDG (PostgreSQL Global Development Group).

El nombre hace referencia a los orígenes del proyecto como la base de datos
"post-Ingres", y los autores originales también desarrollaron la base de datos Ingres.

El proyecto post-ingres pretendía resolver los problemas con el modelo de base
de datos relacional que habían sido aclarados a comienzos de los años 1980. El
principal de estos problemas era la incapacidad del modelo relacional de
comprender "tipos", es decir, combinaciones de datos simples que conforman una
única unidad. Actualmente estos son llamados objetos. Se esforzaron en introducir
la menor cantidad posible de funcionalidades para completar el soporte de tipos.
Estas funcionalidades incluían la habilidad de definir tipos, pero también
la habilidad de describir relaciones - las cuales hasta ese momento eran
ampliamente utilizadas pero mantenidas completamente por el usuario. En Postgres
la base de datos «comprendía» las relaciones y podía obtener información de
tablas relacionadas utilizando reglas. Postgres usó muchas ideas de Ingres
pero no su código.
