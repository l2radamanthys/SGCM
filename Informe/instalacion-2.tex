\section{Instalar Django}

Puedes bajarte Django desde el siguiente enlace \url{https://www.djangoproject.com/download/1.3.7/tarball/}
\footnote {la version 1.3.7 no es la ultima version disponible a la hora de crear
este informe estaba por la 1.6.2 ya que Django se actualiza constantemente.}
te descargara un paquete llamado Django-1.3.7.tar.gz lo descomprimes en
algun directorio luego abres la Terminal y te posicionas sobre el directorio
donde descomprimiste y ejecutas:

\begin{lstlisting}[style=consola]
    $ python setup.py install 
\end{lstlisting}
\vspace{0.1cm}

Sino mediante el instalador de Paquetes de Python de manera mas automatica escribes
en la terminal

\begin{lstlisting}[style=consola]
     pip install django==1.3.7
\end{lstlisting}
\vspace{0.1cm}

Con esto ya tendremos instalado Django.

\section{Instalando el Resto de Las Dependencias}

Ademas de Django en el Proyecto se utilizaron otras Librerias de Python las cuales
algunas vienen instaladas y Otras Requieren ser instaladas de manera similar
a como instalamos Django.

\subsection{psycopg2}

psycopg2 es un adaptador de base de datos PostgreSQL para el lenguaje de
programación Python. psycopg2 fue escrito con el objetivo de ser muy pequeño
y rápido y estable. 

psycopg2 es diferente del otro adaptador de base de datos, ya que fue diseñado
para aplicaciones en gran medida de subprocesos múltiples que crean y destruyen
un montón de cursores y hacen que un número notable de inserciones o
actualizaciones concurrentes. psycopg2 también proporcionan operaciones
asincrónicas completos y apoyo a las bibliotecas de co-rutinas. 

Para instalar descargue el precompilado desde \url{http://www.stickpeople.com/projects/python/win-psycopg/}
Ejecutelo con permisos de administrador, nos pedira que selecionemos la version
de python con que se instalar.

\subsection{ReportLab}

ReportLab es la ultra-robusto motor de código abierto a prueba de tiempo para
la creación de documentos PDF y gráficos vectoriales personalizado. Escrito en
Python, ReportLab es rápido, flexible y una plataforma cruzada.
 
Proporciona un completo conjunto de herramientas de programación para la
creación de documentos y gráficos complejos. Ofrecemos una serie de componentes
 de forma gratuita y de código abierto, además de un paquete comercial con
características adicionales.

Para Instalar descargue el instalado desde \url{http://www.reportlab.com/software/installation/}
y proceda de manera similar a como hizo con la instalacion de psycopg2.


\subsection{easy\_thumbnails}

Easy\_Thumbnails es Una potente aplicación thumbnailing \footnote{Cuando hablamos de thumbnails 
nos referimos a las diferentes miniaturas que son versiones en distintos tamaños
 de una imágen y son usadas para ayudar a su organización y reconocimiento.},
pero fácil de implementacion para Django.

Para Instalar solo ejecute el siguiente comando en terminal, no Se necesita
configurar nada en el proyecto el mismo esta previamente configurado.

\begin{lstlisting}[style=consola]
    pip install easy-thumbnails
\end{lstlisting}
\vspace{0.1cm}

\subsection{django\_extensions}

Django\_Extensions es una coleccion de Extensiones (utilidades) Personalizadas de
diferentes autores no relacionados con el Proyecto Django, para extender las
capacidades del Framework.

Para Instalar solo ejecute el siguiente comando en terminal \footnote{Importante, no todas las funcionalidades
estan soportadas en Windows, pero en cuanto al proyecto no hay problemas.}

\begin{lstlisting}[style=consola]
     pip install django-extensions
\end{lstlisting}
\vspace{0.1cm}


\subsection{django\_cron}

Django-cron permite ejecutar código de Django de manera recurrente para el
seguimiento y ejecución de las tareas. En este caso no es Necesario Instalar
Nada, viene junto con el Codigo Fuente del Proyecto. Igualmente si tiene curiosidad
puede visitar la pagina del proyecto \url{https://github.com/Tivix/django-cron}

\subsection{Descargar e Instalacion de mod\_wsgi}

 Asumiendo que ya  tienes instalado Python y Apache, solo debes descargar el paquete
 libapache2-mod-wsgi ,la ultima version de mod\_wsgi se puede descargar desde su
 pagina oficial \url{https://code.google.com/p/modwsgi/} descargaran un archivo
 similar a "mod\_wsgi-win32-ap22py27-3.3.so" la version que descarguen de mod\_wsgi
 depende como se ve, de la plataforma asi como de la version de python que
 correrar en el servidor. luego por cuestiones de practicidad renombraremos
 el archivo de la siguiente manera:

\begin{lstlisting}[style=consola]
    mod_wsgi-win32-ap22py27-3.3.so -> mod_wsgi.so
\end{lstlisting}
\vspace{0.1cm}

Realizado dicho cambio copiamos el modulo dentro de la siguiente carpeta:
APACHE\_FOLDER \textbackslash modules \textbackslash APACHE\_FOLDER vendria a
ser el directorio donde tenemos la instalacion de WAMP en mi caso es:
C:\textbackslash Apache.

\subsection{Cargando el Modulo en Apache}

Una vez que el módulo de Apache ha sido instalado en el directorio de módulos
de su instalación de Apache, todavía es necesario configurar Apache para cargar
el módulo en realidad.

Abrimos el archivo "httpd.conf" y agregamos la siguiente linea en el mismo
punto donde se cargan el resto de los modulos. \footnote {El archivo httpd.conf
esta en la siguiente ruta en el caso de mi instalacion:
C:\textbackslash Apache \textbackslash conf \textbackslash httpd.conf}

\begin{lstlisting}[style=consola]
    LoadModule wsgi_module modules/mod_wsgi.so
\end{lstlisting}
\vspace{0.1cm}

Con todo esto echo solo tenemos que reiniciar el servidor Apache, en nuestro
 caso clic en el icono en la barra de notificaciones luego las opciones
 Apache->Service->Reiniciar Servicio. 

\subsection{Configuracion del Proyecto}

Bueno Ahora solo tenemos que crear un alias en Apache \footnote{Para mayor
informacion de como crear alias en Apache consulte
\url{http://httpd.apache.org/docs/2.2/urlmapping.html}} para nuestra carpeta
donde colocaremos en mi caso la carpeta destino sera:

\begin{lstlisting}[style=consola]
	C:\Servidor\SGCM
\end{lstlisting}
\vspace{0.1cm}

SGCM es la carpeta contenedora del proyecto, y el alias que usaremos sera:

\begin{lstlisting}[style=consola]
	/sgcm/ 
\end{lstlisting}
\vspace{0.1cm}

tendremos que agregar las siguientes lineas al final del archivo
httpd.conf de apache.

\begin{lstlisting}[style=HTML]
Alias /sgcm/ "C:/Servidor/SGCM/" 
WSGIScriptAlias /sgcm "C:/Servidor/SGCM/handle.wsgi" 

<Directory "C:/Servidor/SGCM">
    Options Indexes FollowSymLinks MultiViews
    AllowOverride all
    Order allow,deny
    Allow from all
</Directory>
\end{lstlisting}
\vspace{0.1cm}



Hay un número de maneras en que una aplicación WSGI organizada por mod\_wsgi
\footnote{Puede consultar \url{https://code.google.com/p/modwsgi/wiki/QuickInstallationGuide}
si desea explorar otras opciones de configuracion.}
puede montarse contra una URL específica. Estos métodos son similares a cómo se
podría configurar las aplicaciones CGI tradicionales.

El principal enfoque implica declarar explícitamente en el archivo de
configuración principal de Apache el punto de montaje URL y una referencia al
archivo de comandos de aplicaciones WSGI. En este caso, el mapeo se fija,
con cambios sólo ser capaz de ser hecho mediante la modificación de la
configuración principal de Apache y reiniciar Apache.

Al utilizar mod\_cgi para alojar aplicaciones CGI, esto se haría mediante la
directiva ScriptAlias. Para mod\_wsgi, la directiva en su lugar se
llama WSGIScriptAlias.

\begin{lstlisting}[style=consola]
WSGIScriptAlias /wsgi "C:/Servidor/SGCM/handle.wsgi" 
\end{lstlisting}
\vspace{0.1cm}

Esta directiva s\'olo puede aparecer en los principales archivos de configuración
de Apache. La directiva se puede utilizar en el ámbito del servidor, pero
normalmente se coloca en el contenedor VirtualHost para un sitio en particular.
No se puede utilizar en cualquiera de las directivas de contenedores ubicación,
directorios o archivos, ni puede ser utilizada dentro de un archivo ''.httaccess''.

El primer argumento de la directiva WSGIScriptAlias debe ser el punto de montaje
URL para la aplicación WSGI. En este caso, la URL no debe contener una barra
diagonal. La única excepción a esto es si la aplicación WSGI es para ser
montado en la raíz del servidor web, en cuyo caso / sería utilizado.

El segundo argumento de la directiva WSGIScriptAlias debe ser una ruta absoluta
para el archivo de comandos de aplicaciones WSGI. Es en este archivo que la
muestra de código de la aplicación WSGI debe colocarse.

Tenga en cuenta que una ruta absoluta debe ser utilizado para el archivo de
comandos de aplicaciones WSGI suministrado como segundo argumento. No es posible
especificar una aplicación por sí sola Python nombre de módulo. Una ruta de
acceso completa se utiliza para una serie de razones, la principal de las
cuales por lo que todos los controles de acceso de Apache todavía pueden aplicarse
para indicar que en realidad puede acceder a la aplicación WSGI.

Porque se aplicarán los controles de acceso de Apache, si la aplicación WSGI se
 encuentra fuera de los directorios que ya están configurados para ser accesible a
  Apache, habrá que decirle a Apache que los archivos dentro de ese directorio se
  pueden utilizar. Para ello se debe utilizar la directiva Directory.

Hasta aqui tenemos mod\_wsgi y nuestro directorio listo, ahora probaremos que
todo va bien para ello dentro del directorio crearemos un archivo llamado
"handle.wsgi" que tendra el siguiente contenido:

\begin{lstlisting}[style=Python]
# -*- coding: utf-8 -*-

import os, sys
import django.core.handlers.wsgi

sys.path.append('C:/Servidor/SGCM')
sys.path.append('C:/Servidor')

os.environ['DJANGO_SETTINGS_MODULE'] = 'settings'

application = django.core.handlers.wsgi.WSGIHandler()

\end{lstlisting}
\vspace{0.1cm}


Con esto nuestro servidor de aplicacion ya deberia funcionar aunque como veran
no se cargan los archivos estaticos como imagenes y hojas de estilo por lo
que necesitamos agregarlo.

%%%


