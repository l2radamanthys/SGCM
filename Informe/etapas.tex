\chapter{Etapas del desarrollo de Proyecto}

\section{Elección del Tema}

Al momento de elegir el tema del seminario se me presento un numero de 
problemas ya que primero no sabia que area se iba a aplicar, queria realizar 
algo distinto, y bueno queria algo que presente un reto a nivel programacion 
, ahora que lo pienso el haber querido esto ultimo me termino saliendo caro por
el tiempo que termine ocupando para terminar el proyecto. 


\subsection{Eleccion de Tecnologia}

Al momento de elegir no tenia casi experiencia desarrollando aplicaciones web 
por lo que al menos ya estaba decidido que iba a ser web la aplicacion, digamos 
fue mas una cuestion de moda, en esos dias habia empezado el boom de la web 2.0
por lo que fue mas por una moda.


\subsection{Eleccion del Tema y Funcionalidades}

Se que suena repetitivo pero bueno digamos fue la parte mas dificil a la hora 
de decidir que queria hacer, aunquen bien el tema elegido, mmm digamos que 
salio de la galera por una interconsulta con un posible cliente que al final no 
llego a mas que intercambiar un par de emails, pero fundo las bases y me 
parecio era una buena idea, consulte un poco los seminarios anteriores aunque 
habia temas relacionados el desarrollo no era muy extensos ni aplicado 
exactamente al area que pretendia.

Habia desarrolladas aplicaciones dedicadas a la facturacion, atencion, etc. Mi 
aplicacion trataba de juntar dos partes el seguimiento o evolucion del paciente
lo cual se registra mediante el documento de Historia Clinica, lo que era bueno 
por que al menos no iba a reinventar la rueda. Llegado el momentos de 
presentar el tema a mi director de tesis me dijo que era poco, mmm supongo que 
fue por que no supe explicarlo bien, en fin habia que agregar algo mas y en 
esos dia tube que asistir a interconsulta medica, cuando entre en el 
consultorio note que la secretaria estaba con una hoja de calculos en MS Excel 
registrando y comprobando los tunos para el dia, de ahi me surgio que para 
complementar la aplicacion podria desarrollar una funcionalidad que permita 
realizar esa tarea de un modo mas eficiente, con ello el tema y las 
funcionalidades quedaron definidas. 


\section{Análisis de Requisitos y Búsqueda de Información}

Luego de elegido el tema de
INCOMPLETO

\section{Primera Aproximación}

INCOMPLETO

\section{Aproximación Final}

INCOMPLETO

\section{Funcionalidades Incluidas y Descartadas}

El nombre del tema o las funcionalidades principales elegidas no dice mucho en 
si sobre que se iba a desarrollar sobre todo en la parte de historia clinica 
ya que esta area abarcava un gran numero de posibles estudios que podrian ser 
incluidos en este documento los cuales algunos son muy especificos del area
de estudio, por ejemplo un odontograma solo seria de interes para un medico
odontologo y en si mucha relevancia no tendria al momento de tratar otras 
enfermedades, por lo que habia que definir que estudios se deberian incluir 
como minimo y cuales se descartarian \footnote{Al menos en esta version, tengase en cuenta que 
esto es un software desarrollado con fines academicos aunque es factible el 
desarrollo de nuevos modulos, su implementacion y aplicacion en casos reales.}.


\subsection{Funcionalidades Incluidas}

Luego de revisar los modelos en papel en su mayoria de diferentes organismos
medicos que consulte se determino que se incluirian los siguientes estudios:

\begin{Itemize}
    \item Habitos Toxicos.
    \item Antecedentes Perinatales \footnote{Hace Referencia a antecedentes del nacimiento}
    \item Grupo Familiar \footnote{Esto no es un Examen en si sino mas bien que sirve para 
        consulta en caso del que el paciente posea enfermedades Hereditarias.}.
    \item Almacenamiento de Imagenes Relacionadas con Diferentes Estudios.
    \item Examen de Cabeza.
    \item Examen de Cuello.
    \item Examen Signos Vitales.
    \item Examen de Piel Faneras y Tejido SubCutaneo.
    \item Examen Osteo Articular.
    \item Examen Sistema Respiratorio.
    \item Examen del aparato Cardiovascular.
\end{itemize}

Adicionalmente tambien se registran las Interconsultas Medicas y los Medicamentos
que fueron recetados al paciente.


\subsection{Funcionalidades no Incluidas}

Si tubiese que nombrar que estudios no se implementaron no terminaria mas aqui
hablando del modulo de Historia Clinica, en esta session mas que nada contemplo 
funcionalidades que en un principio estaban previstas \footnote{las mismas no 
estaban especificadas por escrito en el anteproyecto sino mas bien son 
caracteristicas que tenia pensadas} pero a medida que el desarrollo del sistema 
avanzo perdieron importancia por su aplicacion o por cuestiones de falta de 
tiempo, aqui emnumero algunas de las funcionalidades que se descartaron:

\begin{Itemize}
    \item Gestion de Turnos - Definir Especialidad del Medico para atencion.
    \item Gestion de Turnos - Especificar Habitacion del Consultorio para el Turno.
    \item Historia Clinica - Subir Archivos.
    \item Historia Clinica - Analisis de Laboratorios.
    \item Historia Clinica - Conectar con el Sistema Vademecum.
\end{itemize}

No son todas pero son las mas importentes.


\section{Seguimiento y desarrollo del Proyecto}

El seguimiento y desarrollo del proyecto digamos no fue continuo fue mas algo 
bastante accidentado si podria decirse, y bueno hasta la fecha solo habia 
programado pequeñas aplicaciones, enfrentaba el problema de que me topaba con 
una metodologia diferente de lo que manejaba, a la hora de encarar lo 
relacionado con la programacion.

INCOMPLETO


