\chapter{Elección de la Tecnología}

\section{¿Por que Web y no Desktop?}

Una aplicacion Desktop (tambien llamada de Escritorio) es aquella que requiere
ser instalada en el Ordenador (PC) del Usuario, y que es ejecutada directamente
por el sistema operativo, ya sea Microsoft Windows, GNU/Linux, Mac OS, etc.

Algunos Ejemplos de Estas Aplicaciones:

\begin{itemize}
    \item Winamp
    \item Adobe Photoshop
    \item iTunes
    \item Microsoft Oficce (Word, Excel, Power Point. etc)
\end{itemize}

Aunque suelen ser mas robustas y estables que las aplicaciones Web Presentan
varios inconvenientes tales como:

\begin{itemize}
    \item Su acceso solo se limita al ordenandor donde fue instalada.
    \item La Aplicacion es dependiente del Sistema Operativo que utilice el ordenador, aunque
        existen programas Multiplataforma no aseguran una compatibilidad completa.
    \item Requieren una instalacion personalizada
    \item En caso de Actualizaciones requieren que estas se hagan de forma manual en cada ordenador
        donde se instalo la Aplicacion.
    \item Suelen Tener requerimientos especiales de Software y librerias para poder funcionar.
\end{itemize}


Una aplicacion Web, es aquella que solo requiere ser instalada en un Servidor, su 
ejecucion requiere unicamente Disponer de un ordenador con conexíon a internet
y un navegador en contraparte de las Desktop que requiere que se instale en cada 
ordenador donde se pretende usar.

Por lo cual brinda una serie de ventajas tales como:

\begin{itemize}
    \item Portabilidad, se ejecuta desde cualquier ordenador que posea coneccion a internet sin
        depender de Software adicional, Plataforma y/o Sistema Operativo.
    \item La informacion que se maneja es multiusuario por lo que son especialmente utiles para
        desarrollar aplicaciones multiusuarios basadas en compartir informacion.
    \item Consumen muy pocos recursos, por lo que el usuario no necesita tener un ordenador con 
        grandes prestaciones para trabajar con ellas.
    \item Son faciles de Actualizar y mantener.
    \item Se pueden utilizar en miles de equipos sin limitacion y restriccion alguna.
    \item Su funcionalidad es independiente del Sistema Operativo Instalado en el ordenador del 
            usuario.
    \item No hay problemas de incompatibilidad de Version de software ya que los usuarios trabajan 
        con la misma version.
\end{itemize}
En resumen el Sistema por sus caracteristicas podria haberse implementado como un sistema
Deskop pero se ubiesen perdido las caracteristicas deseadas del mismo.
