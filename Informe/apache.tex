\section{Apache}

El servidor HTTP Apache es un servidor web HTTP de código abierto, para plataformas
Unix (BSD, GNU/Linux, etc.), Microsoft Windows, Macintosh y otras, que implementa
el protocolo HTTP/1.12 y la noción de sitio virtual. Cuando comenzó su desarrollo en 1995 se basó inicialmente
en código del popular NCSA HTTPd 1.3, pero más tarde fue reescrito por completo. Su nombre se
debe a que Behelendorf quería que tuviese la connotación de algo que es firme
y enérgico pero no agresivo, y la tribu Apache fue la última en rendirse al que pronto se
convertiría en gobierno de EEUU, y en esos momentos la preocupación de su grupo era que
llegasen las empresas y "civilizasen" el paisaje que habían creado los primeros ingenieros de internet.
Además Apache consistía solamente en un conjunto de parches a aplicar al servidor de NCSA.
En inglés, a patchy server (un servidor "parcheado") suena igual que Apache Server. 

El servidor Apache se desarrolla dentro del proyecto HTTP Server (httpd) de la
Apache Software Foundation.

Apache presenta entre otras características altamente configurables, bases de
datos de autenticación y negociado de contenido, pero fue criticado por la falta
de una interfaz gráfica que ayude en su configuración.\\[0.2cm]

\section{mod\_wsgi}

mod\_wsgi es un módulo de Apache que provee una interfaz WSGI para correr
 Web en Python sobre Apache. Y esto, es todo lo que necesitas para que tus
 archivos *.py se ejecuten por medio de un navegador Web.

\subsection{WSGI}

WSGI es el acronomico de Web Server Gateway Interface que es una especificacion
para una simple y universal interfaz entre una aplicacion web (en nuestro caso
una aplicacion escrita en Django) y un servidor Web para el lenguaje de programacion
Python. Es un estandar de Python el cual se describe con detalle en la PEP 333
\footnote{\url{http://www.python.org/dev/peps/pep-0333/}}



