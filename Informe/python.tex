\section{Python}

Django esta escrito puramente en Python, por lo que Obiamente Necesitaremos Instalar
Python. Python es un lenguaje de programación interpretado cuya filosofía hace
 hincapié en una sintaxis muy limpia y que favorezca un código legible.
 
Se trata de un lenguaje de programación multiparadigma, ya que soporta
orientación a objetos, programación imperativa y, en menor medida, programación
funcional. Es un lenguaje interpretado, usa tipado dinámico y es multiplataforma.

Es administrado por la Python Software Foundation. Posee una licencia de código
abierto, denominada Python Software Foundation License,1 que es compatible con
la Licencia pública general de GNU a partir de la versión 2.1.1, e
incompatible en ciertas versiones anteriores.

Python es un lenguaje de programación multiparadigma. Esto significa que más
que forzar a los programadores a adoptar un estilo particular de programación,
permite varios estilos: programación orientada a objetos, programación imperativa
y programación funcional. Otros paradigmas están soportados mediante el uso
de extensiones.

Python usa tipado dinámico y conteo de referencias para la administración
de memoria.

Una característica importante de Python es la resolución dinámica de nombres;
 es decir, lo que enlaza un método y un nombre de variable durante la ejecución
  del programa (también llamado enlace dinámico de métodos).
  
Otro objetivo del diseño del lenguaje es la facilidad de extensión. Se pueden
escribir nuevos módulos fácilmente en C o C++. Python puede incluirse en
aplicaciones que necesitan una interfaz programable.

Aunque la programación en Python podría considerarse en algunas situaciones
hostil a la programación funcional tradicional del Lisp, existen bastantes
analogías entre Python y los lenguajes minimalistas de la familia Lisp como
puede ser Scheme.

\subsection{La Filosofia detras de Python}
Los usuarios de Python se refieren a menudo a la Filosofía Python que es bastante
análoga a la filosofía de Unix. El código que sigue los principios de Python de
legibilidad y transparencia se dice que es "pythonico". Contrariamente, el
código opaco u ofuscado es bautizado como "no pythonico" ("unpythonic" en inglés).

Estos principios fueron famosamente descritos por el desarrollador de Python Tim
Peters en El Zen de Python, algunos de ellos son:

\begin{itemize}
    \item Bello es mejor que feo.
    \item Explícito es mejor que implícito.
    \item Simple es mejor que complejo.
    \item Complejo es mejor que complicado.
    \item Plano es mejor que anidado.
    \item Disperso es mejor que denso.
    \item La legibilidad cuenta.
    \item Los casos especiales no son tan especiales como para quebrantar las reglas.
    \item Aunque lo práctico gana a la pureza.
    \item Los errores nunca deberían dejarse pasar silenciosamente.
    \item A menos que hayan sido silenciados explícitamente.
    \item Frente a la ambigüedad, rechaza la tentación de adivinar.
    \item Debería haber una y preferiblemente sólo una manera obvia de hacerlo.
    \item Aunque esa manera puede no ser obvia al principio a menos que usted sea holandés.15
    \item Ahora es mejor que nunca.
    \item Aunque nunca es a menudo mejor que ya mismo.
    \item Si la implementación es difícil de explicar, es una mala idea.
    \item Si la implementación es fácil de explicar, puede que sea una buena idea.
    \item Los espacios de nombres (namespaces) son una gran idea ¡Hagamos más de esas cosas!
\end{itemize}


\subsection{Baterias Incluidas}

Python tiene una gran biblioteca estándar, usada para una diversidad de tareas.
Esto viene de la filosofía "pilas incluidas" ("batteries included") en referencia
a los módulos de Python. Los módulos de la biblioteca estándar pueden mejorarse por
módulos personalizados escritos tanto en C como en Python. Debido a la gran
variedad de herramientas incluidas en la biblioteca estándar, combinada con la
habilidad de usar lenguajes de bajo nivel como C y C++, los cuales son capaces
de interactuar con otras bibliotecas, {\bfseries Python es un lenguaje que combina su clara
sintaxis con el inmenso poder de lenguajes menos elegantes}.

\subsection{Implementaciones}

En la actualidad existen diversas implementaciones de Python

\begin{itemize}
    \item {\bfseries CPython} es la implementación original, disponible para varias plataformas en el sitio oficial de Python.
    \item {\bfseries IronPython} es la implementación para .NET
    \item {\bfseries Stackless Python} es la variante de CPython que trata de no usar el stack de C \url{www.stackless.com}
    \item {\bfseries Jython} es la implementación hecha en Java
    \item {\bfseries Pippy} es la implementación realizada para Palm \url{pippy.sourceforge.net}
    \item {\bfseries PyPy} es una implementación de Python escrita en Python y optimizada mediante JIT \url{pypy.org}
\end{itemize}
