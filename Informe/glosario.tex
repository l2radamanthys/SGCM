\addcontentsline{toc}{chapter}{Glosario}
{\Huge \textbf{Glosario}}
\\[0.5cm]

Glosario de Terminos poco comunes utilizados en este Informe.
\\[0.1cm]

{\LARGE \textbf{A}}\\[0.1cm]

\textbf{Antecedentes Perinatales}: Informacion referente a la atencion durante 
el embarazo y parto.\\[0.1cm]

\textbf{Aplicacion:} Durante el informe se utilizo el nombre Aplicacion o Sistema
indistintamente para referirse al sistema en cual se basa el presente informe.
\\[0.1cm]

\textbf{Aplicacion Informatica:} En informatica, una aplicacion es un tipo de programa 
informatico dise\~nado como herramienta para permitir a un usuario realizar uno o
diversos tipos de trabajos. Esto lo diferencia principalmente de otros tipos de
programas como los sistemas operativos, las utilidades (que realizan tareas de 
mantenimiento o de uso general), y los lenguajes de programacion (con el cual se 
crean los programas informaticos).
\\[0.5cm]

{\LARGE \textbf{B}}
\\[0.1cm]
\textbf{Baterias Incluidas:} El termino hace referencia al tama\~no de la biblioteca
estandar (conjunto de modulos y librerias con que cuenta Python por defecto) ya que
la instalacion basica incluye librerias para casi todos tipo de tareas las cuales
por ende pueden ser extendidas.
\\[0.5cm]

{\LARGE \textbf{C}}
\\[0.1cm]
\textbf{Consulta Medica:} Se refiere a la cita que el paciente tiene con el expecialista.
\\[0.5cm]

{\LARGE \textbf{D}}
\\[0.1cm]
\textbf{DER:} 
\\[0.1cm]
\textbf{Deploy:}
\\[0.5cm]

{\LARGE \textbf{E}}
\\[0.1cm]
\textbf{Expecialista}: En este informe usamos el termino expecialista para referirnos
a los medicos.
\\[0.5cm]

{\LARGE \textbf{H}}
\\[0.1cm]
\textbf{Historia Clinica}
\\[0.5cm]

{\LARGE \textbf{M}}
\\[0.1cm]
\textbf{MVC:}
\\[0.5cm]

{\LARGE \textbf{N}}
\\[0.1cm]
\textbf{No Pythonico:} 
es lo contrario de Pythonico. Osea codigo fuente es ofuscado o tiene
problemas de legibilidad, es un termino de programacion. 
\\[0.1cm]
\textbf{NoSQL:}
En inform�tica, NoSQL (a veces llamado "no s�lo SQL") es una amplia clase de sistemas
de gesti�n de bases de datos que difieren del modelo cl�sico del sistema de gesti�n de
bases de datos relacionales (RDBMS) en aspectos importantes, el m�s destacado es que no 
usan SQL como el principal lenguaje de consultas. Los datos almacenados no requieren
estructuras fijas como tablas, normalmente no soportan operaciones JOIN, ni garantizan 
completamente ACID (atomicidad, consistencia, aislamiento y durabilidad), y habitualmente 
escalan bien horizontalmente.
\\
Por lo general, los investigadores acad�micos se refieren a este tipo de bases de 
datos como almacenamiento estructurado, t�rmino que abarca tambi�n las bases de datos 
relacionales cl�sicas. A menudo, las bases de datos NoSQL se clasifican seg�n su forma 
de almacenar los datos, y comprenden categor�as como clave-valor, las implementaciones
de BigTable, bases de datos documentales, y Bases de datos orientadas a grafos.
\\[0.5cm]

{\LARGE \textbf{O}}
\\[0.1cm]
\textbf{Ofuscado:} 
La ofuscaci�n se refiere a encubrir el significado de una comunicaci�n haci�ndola m�s 
confusa y complicada de interpretar. En computaci�n, la ofuscaci�n se refiere al acto 
deliberado de realizar un cambio no destructivo, ya sea en el c�digo fuente de un programa
inform�tico o c�digo m�quina cuando el programa est� en forma compilada o binaria, con 
el fin de que no sea f�cil de entender o leer.
\\[0.5cm]

{\LARGE \textbf{P}}
\\[0.1cm]
\textbf{Prescripcion Medica}
\\[0.1cm]
\textbf{Pythonico}
\\[0.5cm]

{\LARGE \textbf{S}}
\\[0.1cm]
\textbf{SQL}
\\[0.1cm]
\textbf{Sistema}
\\[0.1cm]
\textbf{Stack}
\\[0.1cm]
\textbf{Super Usuario}
\\[0.5cm]

{\LARGE \textbf{U}}
\\[0.1cm]
\textbf{UML}
\\[0.5cm]

{\LARGE \textbf{Z}}
\\[0.1cm]
\textbf{Zen de Python} Los usuarios de Python se refieren a menudo a la Filosof�a Python
que es bastante analoga a la filosof�a de Unix. El codigo que sigue los principios de
Python de legibilidad y transparencia se dice que es "pythonico". Contrariamente, el
c�digo opaco u ofuscado es bautizado como "no pythonico" ("unpythonic" en ingl�s). 
Estos principios fueron famosamente descritos por el desarrollador de Python Tim 
Peters en El Zen de Python.
\\[0.5cm]
