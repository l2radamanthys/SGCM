\chapter*{Glosario}
\addcontentsline{toc}{chapter}{Glosario}

%{\Huge \textbf{Glosario}}
%\\[0.5cm]

Glosario de términos y nomenclaturas especiales ordenados en orden alfabético que se utilizaron durante la redacción de este Informe.
\\[0.1cm]

{\LARGE \textbf{A}}\\[0.1cm]

\textbf{Antecedentes Perinatales}: Información referente a la atención durante el embarazo y parto.\\[0.1cm]

\textbf{Aplicación:} Durante el informe se utilizo el nombre Aplicación o Sistema indistintamente para referirse al sistema informático desarrollado, en cual se basa el presente informe.
\\[0.1cm]

\textbf{Aplicación Informática:} En informática, una aplicación es un tipo de programa informático dise\~nado como herramienta para permitir a un usuario realizar uno o diversos tipos de trabajos. Esto lo diferencia principalmente de otros tipos de programas como los sistemas operativos, las utilidades (que realizan tareas de mantenimiento o de uso general), y los lenguajes de programación (con el cual se crean los programas informáticos).
\\[0.5cm]

{\LARGE \textbf{B}}
\\[0.1cm]
\textbf{Baterías Incluidas:} El termino hace referencia al tama\~no de la biblioteca estándar (conjunto de módulos y librerías con que cuenta Python por defecto) ya que la instalación básica incluye librerías para casi todos tipo de tareas y las cuales por ende pueden ser extendidas.
\\[0.5cm]

{\LARGE \textbf{C}}
\\[0.1cm]
\textbf{Consulta Médica:} 
Se refiere a la cita que el paciente tiene con el especialista.
\\[0.5cm]

{\LARGE \textbf{D}}
\\[0.1cm]
\textbf{DER:} DER es el acronomico de Diagrama de Entidad Relación, es una herramienta para el modelado de datos que permite representar las entidades relevantes de un sistema de información así como sus interrelaciones y propiedades.
\\[0.1cm]
\textbf{Deploy:}
En este informe hace referencia al proceso de implementación y todas las actividades necesarias para que el sistema software esté disponible para su uso.
\\[0.5cm]

{\LARGE \textbf{E}}
\\[0.1cm]
\textbf{Especialista}: En este informe usamos el termino especialista para referirnos a los médicos.
\\[0.5cm]

{\LARGE \textbf{M}}
\\[0.1cm]
\textbf{MVC:} son las siglas de Modelo Vista Controlador, es un patrón de arquitectura  de software que separa los datos y la lógica de negocio de una aplicación de la interfaz de usuario y el módulo encargado de gestionar los eventos y las comunicaciones. Para ello MVC propone la construcción de tres componentes distintos que son el modelo, la vista y el controlador, es decir, por un lado define componentes para la representación de la información, y por otro lado para la interacción del usuario. Este patrón de arquitectura de software se basa en las ideas de reutilización de código y la separación de conceptos, características que buscan facilitar la tarea de desarrollo  de aplicaciones y su posterior mantenimiento.
\\[0.5cm]

{\LARGE \textbf{N}}
\\[0.1cm]
\textbf{No Pythonico:} es lo contrario del término Pythonico. Ósea código fuente es ofuscado o tiene problemas de legibilidad, es un término de programación. 
\\[0.1cm]
\textbf{NoSQL:}
En informática, NoSQL (a veces llamado "no sólo SQL") es una amplia clase de sistemas de gestión de bases de datos que difieren del modelo clásico del sistema de gestión de bases de datos relacionales (RDBMS) en aspectos importantes, el más destacado es que no usan SQL como el principal lenguaje de consultas. Los datos almacenados no requieren estructuras fijas como tablas, normalmente no soportan operaciones JOIN, ni garantizan completamente ACID (atomicidad, consistencia, aislamiento y durabilidad), y habitualmente 
Escalan bien horizontalmente.
\\
Por lo general, los investigadores académicos se refieren a las bases de dato SQL como bases de datos con almacenamiento estructurado, término que abarca también las bases de datos relacionales clásicas. A menudo, las bases de datos NoSQL se clasifican según su forma de almacenar los datos, y comprenden categorías como clave-valor, las implementaciones de BigTable, bases de datos documentales, y Bases de datos orientadas a grafos.\\[0.5cm]

{\LARGE \textbf{O}}
\\[0.1cm]
\textbf{Ofuscado:} 
La ofuscación se refiere a encubrir el significado de una comunicación haciéndola más confusa y complicada de interpretar. En computación, la ofuscación se refiere al acto deliberado de realizar un cambio no destructivo, ya sea en el código fuente de un programa informático o código máquina cuando el programa está en forma compilada o binaria, con el fin de que no sea fácil de entender o leer.
\\[0.5cm]

{\LARGE \textbf{P}}
\\[0.1cm]
\textbf{PEPs:} Las PEPs o (Python Enhancement Proposals) son un conjunto de propuestas de mejoras del lenguaje de programación Python.
\\[0.1cm]
\textbf{Prescripción Médica:} (también conocida como receta médica) es el documento legal por medio del cual los médicos legalmente capacitados prescriben la medicación al paciente  para su dispensación por parte del farmacéutico.
\\[0.1cm]
\textbf{Pythonico:} Es una filosofía de los programadores de Python para referirse a la legibilidad del código basándose en las especificaciones de la PEP 8.
\\[0.5cm]

{\LARGE \textbf{S}}
\\[0.1cm]
\textbf{SQL} El lenguaje de consulta estructurado o SQL (por sus siglas en inglés Structured Query Language) es un lenguaje declarativo de acceso a bases de datos relacionales que permite especificar diversos tipos de operaciones en ellas. Una de sus características es el manejo del álgebra y el cálculo relacional que permiten efectuar consultas con el fin de recuperar de forma sencilla información de interés de bases de datos, así como hacer cambios en ellas. En el Informe se utiliza el término más que nada para referirse a los motores de base de datos que lo utilizan.
\\
\textbf{Sistema} Durante el informe se utilizo el nombre Aplicación o Sistema indistintamente para referirse al sistema en cual se basa el presente informe. Aunque el termino sistema se utiliza generalmente para referirse a un objeto complejo cuyos componentes se relacionan con al menos algún otro componente, el cual puede ser material o conceptual.
\\[0.1cm]
\textbf{Super Usuario} Es una cuenta de usuario preconfigurada para facilitar la administración del Sistema.
\\[0.5cm]

{\LARGE \textbf{U}}
\\[0.1cm]
\textbf{UML} El Lenguaje Unificado de Modelado (UML, por sus siglas en inglés, Unified Modeling Language) es un lenguaje de modelado de sistemas software. 
\\
Es un lenguaje gráfico para visualizar, especificar, construir y documentar un sistema. UML ofrece un estándar para describir un "plano" del sistema (modelo), incluyendo aspectos conceptuales tales como procesos de negocio, funciones del sistema, y aspectos concretos como expresiones de lenguajes de programación, esquemas de bases de datos y compuestos reciclados.
\\[0.5cm]

{\LARGE \textbf{Z}}
\\[0.1cm]
\textbf{Zen de Python} Los usuarios de Python se refieren a menudo a la Filosofía Python que es bastante análoga a la filosofía de Unix. El código que sigue los principios de Python de legibilidad y.  Estos principios fueron famosamente descritos por el desarrollador de Python Tim Peters en El Zen de Python que comprende un conjunto de recomendaciones sobre buenas prácticas.
\\[0.5cm]

