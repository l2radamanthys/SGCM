\chapter{Introducción}

El presente trabajo de tesis es para obtener el título intermedio de Computador Universitario perteneciente a la carrera Licenciatura en Análisis de Sistemas (plan 97) de la Universidad Nacional de Salta.\\[0.1cm]

El tema elegido para desarrollar corresponde a un sistema de gestión para consultorios médicos, en el se intento reflejar todos los conocimientos que adquirí durante el cursado de la carrera. En cuanto a las razones para la elección del tema son entre otras, el buscar desarrollar un sistema que por sus dimensiones se proponga como un reto, ya que hasta la fecha solo tenía experiencia en cuanto al desarrollo de pequeñas aplicaciones. 

El área de aplicación quedo definida, por la simple razón que tenía contacto con profesionales del área de la Salud.\\[0.1cm]

En cuanto a tecnología quise implementarlo usando algo distinto del tradicional PHP y MySQL para Web, opte por probar una tecnología que no conocía Python, Django y PosgreSQL, la cuales tenía bastante buenas opiniones por parte de terceros y por suerte no decepciono sobre todo Django, que cambio la forma de pensar que tenia a la hora de encarar un proyecto web.\\[0.1cm]


\section{Objetivos Generales}

El objetivo del proyecto de tesis fue el de diseñar y desarrollar un sistema centralizado para el sector de la Salud, específicamente aplicándose al área de consultorios médicos, permitir un mejor seguimiento y control de la evolución de los pacientes mediante la informatización de los diferentes exámenes y consultas que se le realizan al paciente posibilitando la unificación de su historia clínica. Además de también gestionar la asignación de turnos a los pacientes.\\[0.1cm]

Lo que se pretende es brindar un sistema modular y eficiente que permita su fácil aplicación y además de brindar la posibilidad de modificación tanto para adecuación para casos específicos como extensión de sus funcionalidades.\\[0.1cm]


\section{Resumen}

El \textbf{Sistema Web de Gestión de Consultorio Médicos}, desde ahora el \textbf{Sistema}, está pensado para satisfacer las necesidades de un consultorios médicos o cualquier otra actividad en donde sea necesario almacenar información demográfica de pacientes , historias clínicas,  prescripciones así también como la asignación de turnos.\\[0.1cm] 

Al ser un sistema centralizado, se puede acceder a él desde cualquier navegador web actual que cuente con conexión a Internet, lo que permite entre otras cosas: \\[0.1cm]

Disminuir los tiempos de esperas por parte de pacientes a la hora de solicitar ser atendidos, solo necesita solicitar un turno vía web el sistema automáticamente le asignara una fecha y hora acorde a sus requerimientos. Permite a los médicos manejar más fácilmente su agenda para atención de pacientes.\\[0.1cm]

Mejorar el seguimiento de los pacientes por parte de los médicos, centralizando toda su información ya que con ello el médico podrá monitorear la evolución de su paciente donde sea que se encuentre ya que solo necesitara una PC con conexión a Internet.\\[0.1cm] 


\section{Organización del Informe}

El informe está organizado en 8 capítulos, entre los que se incluye esta reseña que corresponde al \textbf{Capitulo I} la cual es una introducción la cual explica las necesidades que motivaron el desarrollo del mismo y un resumen general de lo que se pretende implementar con el mismo. \\[0.1cm]

En el \textbf{Capitulo II} se refiere en cuanto a la tecnología y razones por las cuales se decidió utilizar.  \\[0.1cm]

El \textbf{Capítulo III} trata sobre las etapas del desarrollo del mismo y las diferentes metas alcanzadas.\\[0.1cm]

El \textbf{Capítulo IV} explica la problemática y el funcionamiento de los actuales sistemas aplicados al área. \\[0.1cm]

El \textbf{Capitulo V} habla sobre cuestiones técnicas relacionadas con el desarrollo, los módulos y funcionalidades que fueron necesarios implementar, y las soluciones que se plantearon. \\[0.1cm]

El \textbf{Capítulo VI} no guía en la implementación del sistema, los requerimientos del mismos y toda la configuración necesaria para lograr una correcta instalación y funcionamiento del mismo.\\[0.1cm]

En el \textbf{Capítulo VII} pongo una sencilla visita y explicación de las diferentes áreas del sistema. \\[0.1cm]

Por último el \textbf{Capítulo VIII} es una conclusión, que analiza el desarrollo del mismo, perspectivas a futuro y como podría evolucionar el sistema.

