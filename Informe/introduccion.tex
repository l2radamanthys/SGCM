\chapter{Introduccion}

\section{Objectivos Generales}


El Objectivo del proyecto de tesis fue el de diseñar y desarrollar un Sistema
Centralizado para el area Salud, especificamente aplicandose al area de
consultorios medicos, permitir un mejor seguimiento y control de la evolucion
de los pacientes mediante la informatizacion de los diferentes examenes y consultas
que se le realizan al paciente permitiendo la unificacion de su historia clinica.
Ademas de tambien gestionar la asignacion de turnos a los pacientes.\\[0.1cm]

Lo que se pretende es brindar un sistema modular y eficiente que permita su
facil aplicacion y ademas de brindar la posibilidad de modificacion tanto para
adecuacion para casos especificos como extension de sus funcionalidades.\\[0.1cm]

\section{Resumen}

El \" Sistema Web de Gestion de Consultorio Medicos \" , desde ahora el Sistema,
esta pensado para satisfacer las necesidades de Consultorios Medicos o cualquier
otra actividad en donde sea necesario almacenar imformacion demografica de Pacientes
, Historias Clinicas, Prescripciones asi tambien como la Asignacion de Turnos.\\[0.1cm]

Al Ser un Sistema Centralizado, se puede accerder al el desde cualquier navegador
web actual que cuente con coneccion a Internet, lo que permite entre otras cosas:\\[0.1cm]

Disminuir los tiempos de esperas por parte de pacientes a la hora de solicitar ser
atendidos, solo necesita solicitar un turno via web el sistema automaticamente
le asignara una fecha y hora acorde a sus requerimientos. Permite a los medicos
manejar mas facilmente su agenda para atencion de pacientes.\\[0.1cm]

Mejorar el Seguimiento de los Pacientes por parte de los medicos , centralizando
toda su informacion ya que con ello el medico puede monitorear la evolucion de
su paciente donde sea que se encuentre ya que solo necesitara una pc con coneccion
a internet.\\[0.1cm]







