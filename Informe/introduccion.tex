\chapter{Introducci\'on}

El presente trabajo de tesis es para obtener el t\'{\i}tulo intermedio de Computador 
Universitario perteneciente a la carrera Licenciatura en An\'alisis de Sistemas (plan 97)
de la Universidad Nacional de Salta.\\[0.1cm]

El tema elegido para desarrollar corresponde a un sistema de gesti\'on para consultorios m\'edicos, 
en el se intento reflejar todos los conocimientos que adquir\'{\i} durante el cursado de 
la carrera. En cuanto a las razones para la elecci\'on del tema son entre otras, el buscar
desarrollar un sistema que por sus dimensiones se proponga como un reto, ya que hasta la 
fecha solo ten\'{\i}a experiencia en cuanto al desarrollo de peque\~nas aplicaciones. 
El \'area de aplicaci\'on quedo definida, por la simple raz\'on que ten\'{\i}a contacto
con profesionales del \'area de la Salud.\\[0.1cm]

En cuanto a tecnolog\'{\i}a quise implementarlo usando algo distinto del tradici\'on PHP 
y MySQL para Web, opte por probar una tecnolog\'{\i}a que no conoc\'{\i}a Python, Django
y PosgreSQL, la cual ten\'{\i}a bastante buenas opiniones por parte de terceros y por 
suerte no decepciono sobre todo Django, que cambio mi forma de pensar a la hora de encarar
un proyecto web.\\[0.1cm]


\section{Objetivos Generales}

El Objetivo del proyecto de tesis fue el de dise\~nar y desarrollar un Sistema Centralizado 
para el \'area Salud, espec\'{\i}ficamente aplic\'andose al \'area de consultorios m\'edicos, 
permitir un mejor seguimiento y control de la evoluci\'on de los pacientes mediante la 
informatizaci\'on de los diferentes ex\'amenes y consultas que se le realizan al paciente 
posibilitando la unificaci\'on de su historia cl\'{\i}nica. Adem\'as de tambi\'en gestionar 
la asignaci\'on de turnos a los pacientes.\\[0.1cm]

Lo que se pretende es brindar un sistema modular y eficiente que permita su f\'acil 
aplicaci\'on y adem\'as de brindar la posibilidad de modificaci\'on tanto para adecuaci\'on 
para casos espec\'{\i}ficos como extensi\'on de sus funcionalidades.\\[0.1cm]


\section{Resumen}

El ``Sistema Web de Gesti\'on de Consultorio M\'edicos", desde ahora el Sistema, esta pensado
para satisfacer las necesidades de Consultorios M\'edicos o cualquier otra actividad en donde
sea necesario almacenar informaci\'on demogr\'afica de Pacientes , Historias Cl\'{\i}nicas, 
Prescripciones as\'{\i} tambi\'en como la asignaci\'on de turnos.\\[0.1cm] 

Al Ser un Sistema Centralizado, se puede acceder al el desde cualquier navegador web actual 
que cuente con conexi\'on a Internet, lo que permite entre otras cosas:\\[0.1cm]

Disminuir los tiempos de esperas por parte de pacientes a la hora de solicitar ser atendidos, 
solo necesita solicitar un turno v\'{\i}a web el sistema autom\'aticamente le asignara una 
fecha y hora acorde a sus requerimientos. Permite a los m\'edicos manejar mas f\'acilmente su
agenda para atenci\'on de pacientes.\\[0.1cm]

Mejorar el Seguimiento de los Pacientes por parte de los m\'edicos, centralizando toda su 
informaci\'on ya que con ello el m\'edico puede monitoria la evoluci\'on de su paciente donde 
sea que se encuentre ya que solo necesitara una PC con conexi\'on a Internet.\\[0.1cm]


\section{Organizacion del Informe}

El informe esta organizado en 6 capitulos, entre los que se incluye esta reseña
que corresponde al \textit{Capitulo I} que es una introduccion la cual explica las necesidades 
que motivaron el desarrollo del mismo y un resumen general de lo que se pretende implementar 
con el mismo. \\[0.1cm]

En el \textit{Capitulo II} se refiere en cuanto a la tecnologia y razones por las 
cuales se decidio utilizar.  \\[0.1cm]

% En revision no se incluir
El \textit{Capitulo III} trata sobre las etapas del desarrollo del mismo y 
las diferentes metas alcansadas.\\[0.1cm]

El \textit{Capitulo IV} Esplica la problematica y el funcionamiento
de los actuales sistemas aplicados al area. \\[0.1cm]

El \textit{Capitulo V} habla sobre cuestiones tecnicas relacionadas con el 
desarrollo, los modulos y funcionalidades que fueron necesarios implementar, y 
las soluciones que se plantearon. \\[0.1cm]

El \textit{Capitulo VI} no guia en la implementacion del sistema, los requerimientos del 
mismos y toda la configuracion necesaria para lograr una correcta instalacion y 
funcionamiento del mismo.\\[0.1cm]

En el \textit{Capitulo VII} pongo una sencilla visita y explicacion de las 
diferentes areas del sistema. \\[0.1cm]

Por ultimo el \textit{Capitulo VIII} es una conclucion, que analiza el desarrollo del mismo, 
perspectivas a futuro y como podria evolucionar el sistema.
