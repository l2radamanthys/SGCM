\chapter{Conclusion y Mejoras}


\section{Resultado}

El Sistemas de Gestion de Consultorios Medico proporciona soporte para Gestion 
de Turnso para pacientes y medicos proveyendo una nueva manera de mejorar la 
comunicacion entre el paciente y el medico atraves de Internet, Permite 
administrar las Historias Clinicas dejando de depender de archivos fisicos y 
con la posibilidad de almacenar los mismos en la nube.\\[0.1cm]

Considero que se alcanzaron casi todos los objectivos planteados y otros no 
planteados en la etapa inicial.


\subsection{Ventajas Percibidas}

Las Ventajas y Desventajas en lo que respeta al sistema fueron expuestas en el 
\textit{Capitulo IV} cuando se realizo una comparacion con el actual sistemas,
aqui se analizan las relacionadas a las herramientas que se utilizaron en su 
desarrollo.

\begin{itemize}
    \item La primera ventaja que encontre fue la velocidad de desarrollo 
        comparando con otras herramientas aunque Python no es un 4GL sino un 
        3GL la facilidad de entendiemiento de su sintaxis hace que el codigo
        sea facilmente entendible y legible lo que permite un mantenimiento 
        sencillo, el codigo en python se asemeja mucho a lo que hacemos cuando 
        escribimos un algoritmo en el papel por lo que la curva de aprendizaje
        si ya manejas algun lenguaje es minima, aprendi python en 5 dias.

    \item Django y el Modelo de desarrollo MVC (Modelo Vista Controlador) 
        aportaron otro extra a la velocidad de desarrollo del sistema ya que 
        solo con un par de lineas era capas de crear vistas facilmente 
        adaptables, ademas de la caracteristica de poder heredar plantillas 
        por lo que en caso que quisiera realizar un cambio en el diseño de la 
        plantilla solo requeria cambiar la plantilla maestra o base sin 
        necesidad de estar modificando una por una todas las plantillas, ni 
        hablar si el codigo hubiese estado mesclado con el HTML como ocurre 
        aveses con PHP por ejemplo.

    \item Otra caracteristica interesante de Django que me ahorro sufrimiento
        fue la definicion de Modelos, cuando trabajas con Django no hace falta
        conocer el motor de Base de Datos y su sintaxis, no te debes preocupar 
        por aprender como realizar tal o cual consulta, dejas de pelear con 
        los JOIN de SQL y demas, solo te dedicas a aprender a manejar el 
        ObjectRelationalModel o ORM que forma parte de Django el cual es 
        sencillo de aprender.

    \item Aunque no esta relacionado en si con el desarrollo de manera explicita
        agradesco haber conocido sitios como url{http://www.stackoverflow.com}
        que es un sitio colaborativo donde podes hacer preguntas y/o responderlas
        sobre cuestiones de programacion, instalaciones, errores, etc. Fue una 
        gran ayuda ya que pude solucionar gracias a eso muchas de las 
        dificultades y entender el problema de las mismas de manera rapida.

    \item Aprender a usar un sistema de control de versiones para el codigo 
        fuente como GIT de mi proyecto fue de gran utilidad ya que el desarrollo de esta 
        aplicacion no fue de manera continua sino que variada durante todo el 
        tiempo de desarrollo.
\end{itemize}


\subsection{Desventajas Percibidas}

No todo el desarrollo fue como se esperaba, surgieron una serie de inconvenientes 
o limitaciones relacionadas con la herramienta.

\begin{itemize}
    \item Hacer Deploy \footnote{Implementar un Servidor de Produccion con 
        Apache, Python, Django y PosgreSQl mod\_wsgi} con la herramienta no es 
        tan facil como cuando instalas LAMP, pierdes mucho tiempo intentando 
        configurar el servidor, la documentacion existente sobre la misma es 
        muy poca y normalmente incompleta.

    \item En su mayoria la documentacion sobre las librerias y demas herramientas
        se encuentra escrita en ingles, no lo consideraria en si una desventaja 
        pero lo menciono en este apartado por mi bajo nivel en lo que respecta
        a lectura y comprension de texto en ingles.
\end{itemize}


\section{Futuras Mejoras}

El sistema podria evolucionar de varias maneras, al ser ul sitema diseñado 
mediante plantillas la principal evolucion del mismo es que se podria adaptar 
las interfaces a los navegadores de los dispositivos moviles inteligentes. \\[0.1cm]

Otra mejora comun al sistema, seria que pueda integrarse con otros estudios
como poder registrar analisis de laboratorio, odontogramas, integracion con 
el sistema vademecun para que sea mas sencillo elaborar una receta medica, 
y la posibilidad de importar y/o exportar la historia clinica a formatos 
conocidos como archivos PDF para permitir ser exportado a papel.

