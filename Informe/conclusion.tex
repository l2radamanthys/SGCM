\chapter{Conclusion y Mejoras}


\section{Resultado}

El Sistemas de Gestión de Consultorios Medico proporciona soporte para Gestión de Turnos para pacientes y médicos proveyendo una nueva manera de mejorar la comunicación entre el paciente y el médico atraves de Internet, Permite administrar las Historias Clínicas dejando de depender de archivos físicos y con la posibilidad de almacenar los mismos en la nube.\\[0.1cm]

Considero que se alcanzaron casi todos los objetivos planteados y otros no planteados en la etapa inicial.


\subsection{Ventajas Percibidas}

Las Ventajas y Desventajas en lo que respeta al sistema fueron expuestas en el \textit{Capítulo IV} cuando se realizo una comparación con el actual funcionamiento de la mayoría de las instituciones, aquí se analizan las relacionadas a las herramientas que se utilizaron en su desarrollo.

\begin{itemize}
    \item La primera ventaja que encontré fue la velocidad de desarrollo comparando con otras herramientas aunque Python no es un 4GL sino un 3GL la facilidad de entendimiento de su sintaxis hace que el código
        sea fácilmente entendible y legible lo que permite un mantenimiento sencillo, el código en Python se asemeja mucho a lo que hacemos cuando escribimos un algoritmo en el papel por lo que la curva de aprendizaje si ya manejas algún lenguaje es mínima.

    \item Django y el Modelo de desarrollo MVC (Modelo Vista Controlador) aportaron otro extra a la velocidad de desarrollo del sistema ya que  solo con un par de líneas era capaz de crear vistas fácilmente adaptables, además de la característica de poder heredar plantillas por lo que en caso que quisiera realizar un cambio en el diseño de la plantilla solo requería cambiar la plantilla maestra o base sin necesidad de estar modificando una por una todas las plantillas, ni hablar si el código hubiese estado mesclado con el HTML como ocurre       aveces con PHP por ejemplo.

    \item Otra característica interesante de Django que me ahorro sufrimiento fue la definición de Modelos, cuando trabajas con Django no hace falta conocer el motor de Base de Datos y su sintaxis, no te debes preocupar por aprender cómo realizar tal o cual consulta, dejas de pelear con los JOIN de SQL y demás, solo te dedicas a aprender a manejar el Object Relational Model o ORM que forma parte de Django el cual es         sencillo de aprender.

    \item Aunque no está relacionado en si con el desarrollo de manera explícita agradezco haber conocido sitios como url{http://www.stackoverflow.com} que es un sitio colaborativo donde podes hacer preguntas y/o responderlas sobre cuestiones de programación, instalaciones, errores, etc. Fue una gran ayuda ya que pude solucionar gracias a eso muchas de las dificultades y entender el problema de las mismas de manera rápida.

    \item Aprender a usar un sistema de control de versiones para el código fuente como GIT de mi proyecto fue de gran utilidad ya que el desarrollo de esta aplicación no fue de manera continua sino que variada durante todo el tiempo de desarrollo.
\end{itemize}


\subsection{Desventajas Percibidas}

No todo el desarrollo fue como se esperaba, surgieron una serie de inconvenientes o limitaciones relacionadas con la herramienta.

\begin{itemize}
    \item Hacer Deploy \footnote{Implementar un servidor de producción con Apache, Python, Django y PosgreSQl, mod\_wsgi} con la herramienta no es tan fácil como cuando instalas LAMP, pierdes mucho tiempo intentando configurar el servidor, la documentación existente sobre la misma es muy poca y normalmente incompleta.

    \item En su mayoría la documentación sobre las librerías y demás herramientas se encuentra escrita en ingles, no lo consideraría en si una desventaja pero lo menciono en este apartado por mi bajo nivel en lo que respecta a lectura y comprensión de texto en ingles.
\end{itemize}


\section{Futuras Mejoras}

El sistema podría evolucionar de varias maneras, al ser un sistema diseñado mediante plantillas la principal evolución del mismo es que se podría adaptar las interfaces a los navegadores de los dispositivos móviles inteligentes mediante un diseño responsive. \\[0.1cm]

Otra mejora común al sistema, seria que pueda integrarse con otros estudios como poder registrar análisis de laboratorio, odontograma, integración con  el sistema vademécum para que sea más sencillo elaborar una receta médica,  y la posibilidad de importar y/o exportar la historia clínica a formatos conocidos como archivos PDF para permitir ser exportado a papel.

